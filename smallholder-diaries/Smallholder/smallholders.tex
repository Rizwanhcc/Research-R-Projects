\documentclass[12pt,a4paper]{article}
\usepackage[latin1]{inputenc}
\usepackage{amsmath}
\usepackage{amsfonts}
\usepackage{amssymb}
\usepackage{graphicx}
\author{Rizwan Mushtaq}
\begin{document}
	
\title{Smallholders Financial Diaries in Pakistan}	
\maketitle	
	
\section{Introduction}


The world is digitally connected through Internet and telephone related technologies. A growing number of people are connecting to the others through these mediums. Every connection irrespective of the digital instrument leaves behind digital track. Fintec i.e., the use of new technologies in provision of financial services is also increasing. This venture has two fold objectives. First, availability of timely and accessible financial services to the undeserved, second reduction in cost of doing business. This could also enhance the efficiency of financial providers and cost cutting by using modern technologies. Microfinance services to the poor coupled with new technologies is a powerful combination/tool to reach out to far remote and very poor unbanked communities. Several fintech ventures are underway in developing countries to bring financially excluded into regulated banking system. For instance India is home to the highest unbanked population in the world it has launched ``India Stack'' program with the help of financial and technology sector to increase financial inclusion and to develop a central database. Similarly, in Pakistan ``Telenore easypaisa'' and ``Mobilink Microfinance Bank'', in Bangladesh ``sajida foundation'', ``bKash'' and ``M-Pesa'' in Kenya are the prominent examples of fintech ventures.  fintech enables providers to maintain a comprehensive record of their clients' financial transactions, spending behaviors and financial management skills. This process helps maintaining a big data about financial histories of the clients. That will help providers, donors institutions to design more efficient programs and products to serve undeserved. Moreover, new technologies facilitate ultimate consumers to keep track of their transactions, time saving, quick transfers among others. However, to do so financial institutions must equip undeserved poor about the usage of fintech related devices. As it could become problematic for the unbanked poor to reap the benefits of fintech who can not even read or write. In opposite case, the illiterate poor would remain unbanked or will end up in fraudulent scams and lost privacy. This notion leads to the recognition of the importance of financial literacy for clients and for stable financial markets. Empirical evidence suggests that financial literacy promotes financial inclusion at first, it encourages effective financial market participation while it also improves households financial well being through better decision making. Further, financial literacy results in stable financial markets where lower financial literacy could lead to financial crisis due to inefficient and poor decisions of the participants. Thus financial literacy is equally important at the household level and for the overall financial sector. Therefore, in order to boost effective financial inclusion, building financial skills among masses is essential factor.  


%https://www.dawn.com/news/1334727/adb-approves-20mn-loan-to-increase-credit-access-in-pakistan
%https://tribune.com.pk/story/1416543/adb-approves-20m-loan-micro-small-medium-enterprises/

%In addition to this package, the Asian Development Bank (ADB) has recently approved a loan to expand financial access particularly for MSMEs in Pakistan. An amount of \$20 million provided to Pakistan's largest microfinance bank ``Khushhali Microfinance Bank Limited (KMBL)'' that targets MSMEs, small farmers and poor households. The objective of this loan is to enhance financial access to 30,000 MSMEs from 5,700 by 2020. In addition, this project focuses on the inclusion of women entrepreneurs that makes up 25\% of the recipients.  

%Smallholders are in worse conditions in Pakistan due to illiteracy, insufficient skills and outdated equipment for agriculture.  







\end{document}


%USEFULL SOURCES

%http://www.cgap.org/publications/financial-diaries-smallholder-families
%http://microdata.worldbank.org/index.php/catalog/2555
%http://www.worldbank.org/en/country/pakistan
%http://wasil.org.pk/
%http://www.sajidafoundation.org/ 
%http://partners.wsj.com/metlife/multipliers/videos/?video=vcis62304e
%it is an example of FINTEC initiative in Bangladesh
%https://www.slideshare.net/CGAP/financial-inclusion-for-the-poorest-women-in-pakistan